%XeLaTeX
\documentclass{article}

\usepackage{lscape}


\setlength{\emergencystretch}{150pt}
\usepackage{tocloft}
\cftsetindents{subsubsection}{3em}{7em}
\cftsetindents{subsection}{3em}{6em}
\cftsetindents{section}{3em}{6em}
\usepackage[dvipsnames]{xcolor}
\usepackage{eso-pic,graphicx}
\usepackage[top=41mm, bottom=73mm, outer=50mm, inner=50mm, landscape]{geometry}
\color{White}
\AddToShipoutPictureBG{\includegraphics[width=\paperwidth,height=\paperheight]{semiramis1.jpeg}}

\usepackage{microtype}
\usepackage{fancyhdr}
\pagestyle{fancy}
\fancyfoot[C]{\bfseries\arm{\thepage}}
\fancyhead{}
\usepackage{fontspec}
\usepackage{polyglossia}
\setdefaultlanguage{german}
\setotherlanguages{armenian,russian}

\setmainfont{DejaVuSans}
\setsansfont{DejaVuSans}
\setmonofont{DejaVuSans}
%\newfontfamily{\arm}[Script=Armenian]{DejaVuSans}
%\newfontfamily{\arm}[Script=Armenian]{GHEA Grapalat}%monotonic greek only
%\newfontfamily{\armitalic}[Script=Armenian]{GHEA Grapalat}%monotonic greek only
\newfontfamily{\arm}[Script=Armenian]{DejaVuSans}
\newfontfamily{\armitalic}[Script=Armenian]{DejaVuSans-Oblique}
\newfontfamily{\armenianfontsf}[Script=Armenian]{DejaVuSans}

%\newfontfamily{\arm}[Script=Armenian]{DejaVuSans-Bold}
%\newfontfamily{\armitalic}[Script=Armenian]{DejaVuSans-BoldOblique}
\defaultfontfeatures{Scale=MatchLowercase}

\begin{document}
\bfseries

\renewcommand\thefootnote{{\bfseries{\arabic{footnote}}}}
\let\oldfootnote\footnote
    \renewcommand{\footnote}[1]{\oldfootnote{{\normalsize\bfseries#1}}}
\begin{titlepage} % Suppresses headers and footers on the title page
	\centering % Centre everything on the title page
	%\scshape % Use small caps for all text on the title page

	%————————————————
	%	Title
	%————————————————

	\rule{\textwidth}{1.6pt}\vspace*{-\baselineskip}\vspace*{2pt} % Thick horizontal rule
	\rule{\textwidth}{0.4pt} % Thin horizontal rule
	
	\vspace{1\baselineskip} % Whitespace above the title
	
	{\scshape\Huge \arm{Շաﬕրաﬕ Առասպելը}}
	
	\vspace{1\baselineskip} % Whitespace above the title

	\rule{\textwidth}{0.4pt}\vspace*{-\baselineskip}\vspace{3.2pt} % Thin horizontal rule
	\rule{\textwidth}{1.6pt} % Thick horizontal rule
	
	\vspace{1\baselineskip} % Whitespace after the title block
	
	%————————————————
	%	Subtitle
	%————————————————
	

        {\large \arm{Գրեց Մանուկ Աբեղյան}}
 ‌
	%————————————————
	%	Editor(s)
	%————————————————
        \vspace*{\fill}    

        \vspace{1.0\baselineskip}

        { \arm{Արարատ։ Ամսագիր կրօնական, պատմական, բանասիրական եւ բարոյական գիտելեաց}}
        
	\vspace{1\baselineskip}

        {\small\arm{Էջﬕածին} 1901}
		
	\vspace{0.25\baselineskip} % Whitespace after the title block

        {\scshape\small \arm{Solar Anamnesis Edition}}% Publication year}
    
	{\scshape\footnotesize \arm{CC0 1.0 Universal }} % Publisher
\end{titlepage}
\clearpage
\Large
\paragraph{}
\arm{\emph{Աղբիւրը}։ — Շաﬕրաﬕ և Արայ գեղեցկի մասին պատﬔլիս (Ա․ ժե․) Խորենացին աղրիւր չէ յիշում, ոչ Մար Աբաս և ոչ առասպել կամ երգ։\footnote{Տես «ի բանսն՝ որ զնմանէ» խօսքի և «բանք» բառի բացատրութիւնը Բ․ Գլ․} Բայց, ինչպէս արդէն տեսելենք (Ը․ գլ․), ուրիշ աեղից գիտենք, որ «ի Հայկայ ﬕնչև ցԱրայն Գեղեցիկ, զոր սպանեալ կաթոտն Շաﬕրամ» (Ա․ ե․), ուստի և Արայի պատմութիւնն առնուած է Մար Աբասից։ ԺԷ․ գլխի մէջ ևս, ուր պատմուած է Շաﬕրաﬕ կոտորելն իւր որդիներին, նորա կրիւը Ջրադաշտի հետ և փախուստը Հայաստան և նորա սպանուելն իւր որդի Նինուասից, աղբիւր չէ յիշուած։ Բայց յաջորդ ԺԸ․ գլխի մէջ յայտնում է, որ այդ պատմութիւնն առած է Մար Աբասից։ Այդ գլխի վերնագիրն է․ «Յաղագս թէ \emph{որպէս} հաւաստի նախ լեալ պատերազմ նորա ի Հնդիկս և զկնի \emph{մահուան} (մահ) նորա ի Հայս,»\footnote{№ 1671 Ա․ ձեռագիր․ ﬕւս Ա․ ձեռագիրները «որպես» բաոը չունին․ իսկ №№ 1665, 616 «մահուան» ձեւի փոխանակ ունին «մահ,» որ աւելի ուղիղ ե քերականօրեն։ Շարունակութիւնն ևս աոնում ենք № 1671 ձեռագրից, որին հաﬔմատ են ﬕւս Ա․ ձեռագիրները չնչին տարբերութիւններով։ Տպագիրն աղաւաղուած ե, չունենաշով Շաﬕրաﬕ պատերազմը Ջրադաշտի հետ և սորան յաղթելը։} այսինքն թէ ինչպէս Շաﬕրաﬕ պատերազմը Հնդկաստանում հաւաստաւ առաջ է եղել և յետոյ (եղել է) նորա մահը Հայաստանում։ Ապա գրում է ﬔր պատմագիրը․ «Ունիմ ի մտի և զԿեփաղիոնին, վասն ոչ տալ զﬔզ բազմաց ծաղրել․ զի ասէ ի բազմաց (Տպ․ ի բանից) այլոց՝ նախ յազագս ծննդեանն Շաﬕրամայ, \emph{եւ ապա զպատերազﬓ Շաﬕրամայ ընդ Ջրադաշտի, եւ զյաղթելն ասէ Շաﬕրամայ}, և ապա ուրեﬓ զպատերազﬓ Հնդկաց։ Այլ հաւաստի ﬔզ թուեցաւ որ ի Մարիբաս Կատինայն է քննութիւն քաղդէական մատենից քան զայսոսիկ [այսինքն Կեփաղիոնի պտամածը Շաﬕրաﬕ մասին]․ քանզի ոճով իﬓ ասէ [Մար Աբաս] և զպատճառս պատերազﬕն [Ջրադաշտի հետ, որի հետևանքն եղել է Շաﬕրաﬕ մահը Հայաստանում] յայտնէ․ իսկ առ այսոքիւք և առասպելք աշխարհիս ﬔրոյ զբազմահմուտ ասորին արդարացուցանեն․ աստ ուրեﬓ զմահն ասել Շաﬕրամայ» և այլն։

Խորենացին Կեփաղիոնի մասին և նորա գրածը Շաﬕրաﬕ համար առնում է, հաստաա կարելի է ասել, Եւսեբիոսի քրոնիկոնից, թէպէտ և վերջինիս անունը չէ տալիս։\footnote{«Կեփաղիոնի վիպագրի՝ վասն ասորեստանէաց թագաւորութեանն։» «Ապա ի նոյն յարեալ՝ ասէ [Կեփաղիոն] և զծնունդն Շաﬕրամայ․ եւ զգարաւըշտ մոց (ի) արքայի Բակտրացւոց զպատերազﬔն և զպարտուբեանե ի Շաﬕրամայ․ և զանս բագաւորութեանն Նինայ անս ծբ․ և զվախնան նորա։ Յետ որոյ թագաւորեաշ Շաﬕրամայ ած պարիսպ Բաբեղոնի զայն ձեւ օրինակի, որպէս և բազմաց իսկ ասացեալ է, կտեսիայ և զենոնի և Էրոդոտայ, և այշոց՝ որ յետ նոցա։ Ապա և զգօրաժողոﬖ շինեշ Շաﬕրամայ ի վերայ հնդկաց աշխարհին վիպագրե, և զպարտութիւն նորա և զփախուստ․ և թէ զիա՛րդ ինքնին զիւր որդիսն կոտորեաց, և ինքն ի Նինրայ որդւոյն իւրոյ սպանաւ։» Եւսեբի Կեսարացույ ժամանակակաւք, Վեներիկ 1818 եր․ 90 հտն․} Այստեղ իրօք որ առանց «ոճի» և առանց «պատճառները» յայտնելու բերած է Շաﬕրաﬕ ծընունդն, նորա պատերազմը Ջրադաշտ մոգի հետ և յաղթելը, ապա Շաﬕրաﬕ պաաերազմը Հնդկաստանում և փախչելը․ նորա սպանելն իւր որդիներին և իւր սպանուելն Նինուասից։ Մար Աբասի մէջ այս աﬔնը, թէպէտև Կեփաղիոնից տարբեր, բայց ոճով և պատճառաբանուած է պատմուած եղել (Ա․ ժէ․), ուստի և Խորենացին, Հնդկաց հետ պատերազմը հաւաստի համարելով \emph{առաջ} եղած է դնում, և ﬓացածն առնում է Մար Աբասից։

Այսպէս Խորենացին Շաﬕրաﬕ պատմութիւնն առնում է Մար Աբասից։ Բայց նա Մար Աբասի պատմածը Շաﬕրաﬕ մահուան մասին հաստատելու համար յիշում է թէ Շաﬕրաﬕ մասին առասպելներ եղել են հայոց մէջ։ Դժբախտաբար նա շատ բան չէ բերում այդ առասպելներից, այլ բաւականանում է Շաﬕրաﬕ մահուան կամ վեըջի մասին ﬕայն յիշելով, որովհետև խնդիրն այն է, որ հաստատէ թէ Շաﬕրամը Ջրադաշտի հետ պատերազﬕ մէջ Հայաստան է փախած և այստեղ սպանուած, ինչպէս դնում է Մար Աբաս, հակառակ Կեփաղիոնի, որ նորան Հնդկաց պատերազﬕ մէջ փախած է դնում և այդ պատերազﬕց յետոյ սպանուած։ Այնուհետև ԺՋ․ գլխի պատմութիւնը Խորենացին այնպիսի դարձուածներով է պատմում, որոնցից պարզ երևում է դարձեալ, որ իւր ժամանակ աւանդութիւն եղել է հայոց մէջ թէ Վանայ հնութիւններն ու «ամբարտակ գետոյն» վերագրուած են եղել Շաﬕրաﬕն։ Այս գլուխը Խորենացին, ինչպէս գիաենք, իւր լեզուով և իւր կողﬕց է պատմում, բայց ոճերի և դարձուածների համար օդտւում է դարձեալ Եւսեբիոսի Քրոնիկոնից։ Մենք այդ մասին կանգ առնել չենք ուզում, զի խնդիրն այն չէ թէ Խորենացին ինչպիսի ոճաբանութիւն ունի․ արդեօք նա իւր \emph{սեպհական բառերով ու դարձուածներով} է նկարագրում որևէ բան, թէ այս կամ այն տեղից օգտուելով է անում իւր սեպհական նկարազիրը։ Նկատենք ﬕայն, որ այդ կողﬕց ոճի աղքատութիւնը յատուկ է Խորենացուն։ Նա յաճախ ﬕշտ նոյն ձևով ուրիշներից օգտուելով է նկարագրում, ինչպէս Յովհաննէս կաթուղիկոսը, երբ նա նոյն իսկ իւր ժամանակի անձերն ու դէպքերը նկարագրելիս օգտւում է Խորենացու դարձուածներից։ Բայց ոչ ոք կարող չէ ասել թէ որովհետև Յովհաննէս կաթուղիկոսն, օրինակ, Աշոտ թագաւորին և նորա «յարդարած կարգերը» Խորենացուց օդտուելով է նկարագրում, ուստի և նա սուտ է գրում և պատմութիւնը սարքում է։ Դա ոճի առանձին յատկութիւն և աղքատութիւն է ﬕայն։ Այսպէս և Խորենացին Վանայ նկարագրի մէջ ինչքան էլ Եւսեբիոսից կամ ուրիշներից օգտուած լինի, նորա նկարագրի հիմքը ﬓում է իբրև իրողութիւն․ դա Վանի և նորա հնութիւնների նկարագիրն է, և թէ այդ հնութիւնների շինողը Խորենացու ժամանակ համարուել է Շաﬕրամ։ Բայց ﬔնք դառնանք ﬔր խնդրին։

\emph{Իշտար, Շաﬕրամ, Աստղիկ (Անահիտ)}։ — Երբ ﬕ անգամ ըստ Խորենացու վկայութեան Շաﬕրաﬕ մասին առասպել եղել է Հայոց մէջ, ըստ ինքեան անհաւանական չէ, որ ﬕ ժողովրդական զրոյց եղած լինի Շաﬕրաﬕ և ﬕ հայի, Արայի, յարաբերութեան համար։ Թէպէտև Խորենացու աղբիւրը գրաւոր է, Մար Աբաս․ բայց վերջինիս աղբիւրը կարող է ժողովրդական եղած լինել, ուստի և Շաﬕրաﬕ և Արայի զրոյցը ժողովրդական ծագում ունենալ։ Տեսնենք այդ զրոյցի բովանդակութիւնը, հաﬔմատելով ուրիշ ազգերի մէջ եղած նոյն զրոյցի հետ։ Բայց նախ Շաﬕրաﬕ բնաւորութիւնը։

Շաﬕրաﬓ ըստ հնոց դուստր է Ասորեստանցոց Միւլիտտա դիւցուհու կամ ասորական Դերկետոյի։ Նա նոյն բնաւորութիւնն ունի ինչ որ իւր մայրը, սեմական ազգերի սիրոյ դիցուհին, Դերկետոյ կամ Աստարտէ (Atargatis), բաբելացոց և ասորեստանցոց Իշտարը։

Ասորեստանցոց մէջ այդ սիրոյ դիցուհին ﬕ հզօր հրամանատար, մարտիկ աստուածուհի է, պատերազﬕ դիցուհին նետ֊աղեղով զինուած, այրական ու պատերազմական բնաւորութեամբ, ճակատամարաի և որսի թագուհին։ Բայց Իշտարը ﬕանգամայն և փարթամ պտղաբերութեան, հեշտութեան և զգայական սիրոյ դիցուհին է։ Նա կապուած է Արուսեակ մոլորակի հետ և նորա սիմբոլն է գիշերավարը (Արուսեակն արևը մտնելուց յետոյ)։ Փիւնիկէում և Ասորիքում այս սիրոյ դիցուհու նուիրական թռչունն էր աղաւնին։ Նորա պաշտամունքը Բաբելացոց մէջ ցոփ և անառակ էր․ նոյնպէս բուռն զգայական և անբարոյական էր ասորական Աստարտէի, ըստ Կտեսիասի, Դերկետոյի պաշտամունքը։ Նոյն բնաւորութիւնն ունի և փիւնիկական Աշտարտը որ պաշտւում էր կանաչ բլուրների և սրբազան անտառների մէջ։\footnote{Lehrbuch der Religionsgeschichte, herausg. von P. D. Chantepie de la Saussaye, Leipzig. 1897, 1. եր․ 190 հտ․ 197 հտ․ 225 հտ.} Շատ տեղեր հին ժամանակ եղել են Շաﬕրաﬕ անունով բլուրներ։ Այսպէս Ստրաբոնը գրում է Կապադովկիայի Դիանա քաղաքի համար․ «Քաղաքն շինեալ է ի վերայ ﬕոյ \emph{ի սարահարթ րարձրաւանղակաց անտի, որ կոչին Շաﬕրամայ}։ «(Ստր․ ԺԲ․ 2․ 7)։ Պ․ Գարագաշեանը բերելով այս հատուածն աւելացնում է․ «Տիանայ, որպէս և գաւառն Տիանիտիս՝ կըյիշեցունեն զԱնահիտ․․․ և էր արդարև, ըստ նﬕն Ստրաբոնի, ոչ կարի հեռի ի Տիանայ, անուանի ﬔհեանն Անահտայ Պերասիայ։» «Բայց քաղաք նուիրեալ բուն Անահտայ էր Ջեղա, յորմէ ասէ Ստրաբոն թէ էր ի վերայ բարձրաւանդակի որ կըկոչուէր Շաﬕրամայ։» «Ջեղա, ասէ Ստրաբոն, ունի ﬔհեան անուանի նուիրեալ Անահտայ, այն է դիցն զոր պաշտեն և Հայք։»\footnote{Գարագաշեան, Քնն․ պատմ․ Հայոց Բ․ եր․ 167 հտ․}

Սեմական ազգերի սիրոյ աստուածուհու պաշտամունքը Կիպրոսից կղզիների վրայով անցած էր և Կիւթերա և Սիկիլիա, ուր փիւնիկեան Աշտարտի անբարոյական պաշտամունքը խառնուած և ﬕացած էր յունական Ափրոդիտէի պաշտամունքի հետ։ Այս դիցուհին պաշտւում էր և Հայոց մէջ։ Մեր Աստղիկը, որ համապատասխան են դնում Ափրոդիտէին, համարւում է ասորիներից փոխառութիւն։ Նորա անունն իսկ՝ Աստղիկ, ըստ Հոֆմանի, թարգմանութիւն է ասորերէն Կաուկաբտա (kaukabta) բառի, որ նշանակում է աստղիկ, Արուսեակ (Venus) մոլորակը,\footnote{Gelzer Zur arm. Götterlehre, եր․ 123. 132.} ﬔր այժﬔան ժողովրդական լոյս֊աստղը։ Սակայն ինշքան էլ Աստղիկը, որի պաշտամունքի մասին շատ տեղեկութիւն չունինք, սեմականների սիրոյ դիցուհին լինի՝ մտած հայոց մէջ, հաւաստի յայտնի է, որ հնումը ﬔր Անահտի պաշտամունքը նոյն է եղել, ինչ որ սեմական սիրոյ դիցուհունը։ Անահիտը, թէպէտև իրանական ծագում ունի, կրել է սեմական ազդեցութիւն։ Ստրաբոնը գրում է թէ պարսից բոլոր աստուածները պաշտում են հայերը, մանաւանդ Անահտին, որին զանազան տեղերում և Եկեղիքում ﬔհեաններ են կանգնած, և թէ նորան այր և կին գերիներ են նուիրում․ և այնուհետև աւելացնում է․ Մինչև ցայս վայր չիք տեղի զարմանալոյ․ բայց դիցապաշաութիւն հայոց երթայ անգր ևս, զի սովորութիւն է առ նոսա արանց աւագաց ձօնել դիցն զդստերս իւրեանց կուսանս․ այլ այսչէ ինչ արգել յետ տալոյ զանձինս ի պոռնկութիւն ի ﬔհեանսն Անահտայ զտանել արս որ ոչ խղճիցեն առնուլ ղնոսա ի կնութիւն։\footnote{Գարագաշեան, Քնն․ պատմ․ Հայոց Ա․ եր․ 267 հտ․ Gelzer, Zur arm. Götterl. եր․ 113.} Անահտի այս պաշտամունքը կատարելապէս նոյն է, ինչ որ սեմական սիրոյ աստուածուհունը։ Հայոց մէջ այս անառակ պաշտամունքը գտնելը շատ բնական է, երբ ինկատի ունենանք, որ Հայերը հարաւից և արևմուտքից ոչ ﬕայն սահմանակից են եղել սեմական ազգերին, այլ և Հայաստանի հարաւային և արևմտեան կողﬔրի բնակիչները սկղբնապէս եղել են ասորիք և յետոյ են հայացած։ Այսպէս Անահտի ﬔհեաններ եղել են Եկեղիքում և Տարօնում, ուր, ըստ Ստրաբոնի, բնակիչներն առաջ եղել են ասորիներ։ Տարօնի Աշտիշատում Անահտի հետ եղել է և Աստղիկի պաշտամունքը։ Անահտի ﬕ ուրիշ ﬔհեան եղել է Վանից հարաւ Անձևացեաց մէջ։ Նոյն տեղերում Պաղատոյ լեռան գլխին նաև Աստըղկի պաշտամունքը։\footnote{Մ․ Խորենացու մատենագրութիւնք․ եր․ 294, 301․ «Տուեալ յայնմ տեղւոշե դեղս ախտականս առ ի կատարել զպղծութիւնս ախտից․․․ առեալ ի չաստուածոցն ծրարս թարախածորս ի պատիր ախտիցն, որպես զծրարսն Կիպրիանոսի առ ի պատիր Յուստինեայ կուսին» (եր․ 294)։} Վանայ մօտ Արտաﬔտում ևս Աստղկի պաշտամունքը յայտնի է Թոմա Արծրունուց (եր․ 53 հտ․)։ Այնուհետև Վան քաղաքը, առաջին անգամ Խորենացու մէջ, կոչւում է քաղաք Շաﬕրամայ․ նոյնը յիշում է և Թոմա Արծրունին (եր․ 63․ 240․ 252)․ բացի այդ սա յիշում է Ճուաշ գաւառում Շաﬕրամ բերդ (եր․ 258 հտ․ 281), ամբոցն Շաﬕրամ (եր․ 270)։ Շաﬕրամ գիւղ այժմ Նէմրութ սարի ստորոտում յիշում է Սարգիսեանը իւր տեղագրութեանց մէջ (եր․ 272)։

Շաﬕրաﬕ անունով այս տեղերի կոչուﬓերը հայոց մէջ Խորենացու ազդեցութեամբ մտած համարելու համար ոչ ﬕ հիմք չունինք։ Նոյն իսկ պ․ Խալաթեանն, ինչքան էլ Շաﬕրաﬕ մասին առասպելները Խորենացու սարքածն է համարում, ստիպուած է ասելու, որ Խորենացին Շաﬕրաﬕ զրոյցների համար ընդհանրապէ․ «իբրև ելակէտ ունի (Исходить) Հայաստանի տեղադրական անունները, որոնք կապուած են Ասորեստանի աշխարհակալ թագուհու անուան հետ Ուրիշ խօսքով, Խորենացու ժամանակ եղել են Շաﬕրաﬕ անունով տեղեր և Խորենացին այդ անուններից օգտուելով սարքել է Շաﬕրաﬕ մասին զրոյցներ։ Իսկ այդ տեղագրական անունները յիշուած են, գրում է պ․ Խալաթեանը, «ըստ երևութին, ﬕայն Ը․-Թ․ դարից, այսինքն նոյն ժամանակից, երբ այդ ﬕջոցին զօրացած Արծրունի իշխանների մէջ ձդտուﬓ է առաջ գալիս Ս․ Գրքի ﬕ ակնարկութեան հիման վրայ՝ իրենց ցեղը հանել Ասորեստանի թագաւորներից։» Որքան հասկանում ենք պ․ Խալաթեանին, նա ուղում է ասել, որ այդ տեղերի անունները Շաﬕրաﬕ անուանը կապել են շատ ուշ ժամանակում Արծրունի իշխանները։ Սակայն պ․ Խալաթեանը այդ ասում է լոկ խօսքով։ Բայց, ինչքան էլ Արծրունիք իրենց հանէին Ասորեստանի թադաւորներից, ոչ ﬕ հիմունք չունինք թէ նոքա իրենց նախնի համարած Սենեքերիﬕն, Սանասարին կամ Ադրաﬔլիքին թողած՝ Շաﬕրաﬕ անունով պիտի կոչէին իրենց Վանտոսպը։ Եթէ Շաﬕրաﬕ անունով տեղեր և բլուրներ եղել են հընումը շատ շատ կողﬔրում և նոյն իսկ Կապադովկիայում և Պոնաոսում, և եթէ սեմականների այդ սիրոյ դիցուհին ի թիւս այլոց պաշտուել է և հայոց մէջ, կարող են ի հնուց անտի եղած լինել Շաﬕրաﬕ անունով կոչուած տեղերը Հայաստանի յատկապէս այն կողﬔրում, որոնց բնակիչները ասորիներ են եղել և սեմականներին սահմանակից։ Միւս կողﬕց եթէ ըստ Խորենացու Արծրունիք և Աղձնեաց բդեշխները իսկ ըստ Թոմա Արծրունու և Սասնոյ բնակիչները, իրենց սերած են համարում Ասորեստանցիներից, կարծում ենք, այդ ևս ﬕ պատմական յիշողութիւն է ﬕայն, քանի որ այդ կողﬔրի հայ բնակիչները ծագել են իրօք ասորիներից։ Տեղական ժողոկըրդի մէջ եղել է այդ պատմական յիշողութիւնը, որ յետոյ կարող էր հեշաութեամբ կապուել Ս․ Գրքի մէջ յիշուած Սենեքերիﬕ որդոց հետ։ Սենեքերիմ և իւր որդիքը կարող են յետոյ Ս․ Գրքից մտած լինել, բայց ոչ Ասորեստանցիների ծագման զրոյցը։ Այդ զրոյցը հին պէտք է համարել, ինչպէս հին են նաև Շաﬕրաﬓ ու իւր զրոյցները և ոչ սարքովի բան։ Տեսնենք այդ զրոյցները։

\emph{Իշտարի ու Իզդուբարի եւ Շաﬕրաﬕ ու Արայի առասպե լները}։ — Ասորեստանցոց Իշտարը, որ չամուսնացած աստուածուհի է, իւր տարփանքի համար հոմանիներ է որոնում․ բայց նա հեշտասէր կողﬕ հետ ունի և ﬕ սոսկալի մահաբեր կողմ, որով նա իւր սիրականին մահացնում է և ապա վշտից հետևում է նորան ﬕնչե Սանդարաﬔտի բանղը՝ իւր տարփածուին յարութիւն տալու համար։ Իշտարի առասպելից յայտնի է Իզդուբարի նշանաւոր վէպը։

Հրաշալի գեղեցիկ դիւցազն է Իզդուբար, որին սիրում է Իշտարը։ Դիցուհին ինքն իրեն առաջարկում է գեղեցիկ Իզդուբարին, ﬔծ պարգևներ և իշխանութիւն խոստանալով նորան, որ իւր կամքը կատարէ․ բայց Իզդուբարը ﬔրժում է նորա րոլոր առաջարկութիւնները։ Այդ ժամանակ Իշտարը զայրանում է և սպանում Իզդուբարին։ — Այսպէս սպանում է նա իւր ﬕ ուրիշ տարփածուին ևս Տամմուզին (tammuz), որի մահը ողբում է նա յետոյ և Սանդարաﬔտ է իջնում նորան յարութիւն տալու համար։\footnote{Lehrbuch der Religionsgeschichte, herausg. von P. D. Chantepie de la Saussaye, Leipzig, 1897. 1. եր․ 192, 216․}

Ասորեստանցոցու Բաբելացոց մէջ այս առասպելի պաշտամունքնևս կար իւր արարողութիւններով։ Նոյնը կար և ասորոց մէջ Աթարի (Athar) և Ատտեսի (Attes) համար իսկ Փիւնիկեցոց մէջ Աշտարտի պաշտամունքի հետ ﬕացած էր նորա սիրական Ադոնիսի պաշտամունքը․ Տիւրոսում նոյն իսկ Ադոնիսի յարութեան տօն էին կատարում։

Արդ ի՞նչ է ﬔր Շաﬕրաﬕ և Արայի ղըրոյցը։ Ինչպէս «այրասիրտն այն և կաթոան Շաﬕրամ» նոյն է, ինչ որ իւր մայրը սեմականների սիրոյ աստուածուհին, մարտիկ և տարփագին դիցուհին Իշտար — Աստարտէ — Ատարդատիս կամ Դերկետոյ, նոյնպէս և մէր Շաﬕրաﬕ և Արայի զրոյցը նոյն է, ինչ որ Իշտարի և Իզդուբարի այս վէպը։ Հաﬔմատութիւնը շատ պարզ է։ Մեր Արան նոյնպէս գեղեցիկ է, ինչպէս Իզդուբար, Ադոնիս և ուրիշները։ Ինչպէս Իշտարն Իզդուբարին, նոյնպէս և Շաﬕրաﬓ Արային իշխանութեան խոստուﬓեր է անում իւր կամքը կատարելու համար։ Բայց Արան հաւանութիւն չէ տալիս, ինչպէս և Իզդուբարը։ Իշտարը զայրանում և մահացնում է Իզդուբարին․ այսպէս և Շաﬕրամ տիկինն «ի սաստիկ ցասման լեալ» գալիս կռւում է Արայի դէմ․ և Արան ﬔռնում է այդ սիրոյ պատճառով։ Այնուհեաև ﬔծ մայր Իշտարը սաստիկ ցաւում է և Սանդարաﬔտ է իջնում իւր տարփածուին յարութիւն տալու համար։ Նոյն յարութեան ﬕջադէպը կայ և ﬔր առասպելի մէջ, ﬕայն ﬔր հեթանոս հայերի հաւատալիքով պատմուած, և այս շատ բնական է։ Խորենացու և Անանունի մէջ այդ յարութիւնը քրիստոնէական հայեացքով փոփոխութեան է ենթարկուած, որ նոյնպէս շատ բնական է։

\emph{Առլէզք եւ Արայի յարութիւնը}։ — Յայտնի է, որ ﬔր հեթանոս հայերն ունեցել են, ըստ Եզնիկի «ի շանէ ելեալ» Արալեզք կամ Առլեզք կոչուած ոգիներ, «աներևոյթ զօրութիւնք,» որ պատերազﬕ մէջ ընկած վիրաւոր քաջերին կամ դիւցազներին լիգում և ողջացնում են։ Այդ հաւատալիքը ինչպէս երևում է, շատ հին պիտի լինի, քանի որ Պլատոնի հասարակապետութեան մէջ բերուած \emph{Էր հայի} յարութիւն առնելը, թէպէտ և առանց Արալեզների, ﬔր այդ հաւատքի հետ նոյն ընդհանուր գծերն ունի։ Այստեզ ևս Էրը քաջասիրա է և սպանւում է պատերազﬕ մէջ։ Տասն օրից յետոյ, ﬕնչ ուրիշների դիակները նեխած էին, Էրինը անխախտ և ամբողջ են գանում։ Բերում են տում, և տասներկուերորդ օրը, երբ խարոյկի վրայ էր բարձրացած, կենգանանում է Էրը։ Էﬕնն առաջին անգամ այս առասպելը հաﬔմատութեան բերելով Արա գեղեցկի առասպելի հետ, աւելացնում է թէ Պլատոնի առասպելի իմաստը, «\emph{քաջք անկեալք ի պատերազﬕ}» \emph{յառնեն}, ոչ այլ ինչ է, բայց եթէ բուն հեթանոսական վարդապետութիւն նախնի Հայոց։ Եւ նա աշխատում է Էր հայը նոյնացնել Արա Հայկաղնի հետ։\footnote{Վեպք Հնոյն Հայաստ․ եր․ 146 հտն․} Այս հաւատն այնքան զօրեղ է եղել, որ նոյն իսկ քրիստոնէութեան ժամանակ Դ․ դարում, ըստ Փաւստոս Բիւզանդի (Ե․ դպր․ լզ․ գլ․), չնայելով որ Մուշեղ Մաﬕկոնեանի գլուխը մարﬓից կտրուած էր, բայց «ոչ հաւատային ընտանիք նորա մահուն նորա․․․ իսկ կէսք յառնելոյ ակն ունէին նմա։» Այս պատճառով գլուխը կպցնում են մարﬓին և դնում ﬕ աշտարակի տանիքում, կարծելով թէ «վասն զի այր քաջ էր, Աոլեղք իջանեն և յարուցանեն զդա։»

Այսպէս պատերազﬕ դաշտում իբրև քաջ ընկնում է և Արան․ «գտանեն զԱրայն ﬔռեալ ի մէջ \emph{քաջամարտկացն, եւ հրամայէ ղընել զնա ի վերնատանն ապարանից}։ Իսկ ի գրգռել ﬕւսանգամ զօրացն Հայոց ի մարտ պատերազﬕ ընդ տիկնոջն Շաﬕրամայ՝ քինախնգիր լինել մահուան Արայի, ասէ․ \emph{հրամայեցի աստուածոցն իմոց լեզուլ զվէրս նորա եւ վենղանասցի}։» Եւ Շաﬕրաﬓ իրօք սպասում է, որ Արան պիտի կենդանանայ․ «\emph{Միանզամայն եւ ակն ունէր} ղիւթութեամբ վըհկութեան իւրոյ \emph{վենդանացուցանել զԱրայ}, ցնորեալ ի տռփական ցանկութենէն։» Քրիստոնեայ մատենագիրը, որ այդպիսի աստուածների զօրութեան չէր հաւատում, Շաﬕրաﬕ ձգտուﬓ՝ աստուածների ձեռով Արային կենդանացնելու բացատրում է «դիւթութեամբ վհկութեան։» Միւս կողﬕց նա չէր կարող երբէք հաւատալ, որ աստուածները կամ Շաﬕրամ իւր կախարդութեամբ կենդանացրած լինին Արային․ ուստի և պիտի գրէր թէ «Նեխեցաւ դի նորա, հրամայեաց ընկենուլ ի վիհ ﬔծ և ծածկել։» Այստեղ եթէ վերջացնէր, առասպելը թերի կըﬓար, զի հայերը գրգռուած էին և նոքա հանդարտում են, երբ իմանում են, որ Արան յարութիւն է առել․ «Եւ այսպէս համբաւեալ զնմանէ ի վերայ երկրիս Հայոց․ և հաւանեցուցեալ զաﬔնեսեան դադարեցուցանէ զխազﬓ։» Ուստի և Արայի յարութիւն առնելու պատմութիւնը պէտք էր պահել, ﬕայն ﬕ ձևով բացատրած։ Եւ նա բացատրած է շատ պարղ կերպով․ «\emph{Ջﬕ ոﬓ ի հոմանեաց իւրոց զարդարեալ ունելով ի ծածուկ}, համբաւէ զնմանէ այսպէս․ լիզեալ աստուածոցն զԱրա և կենդանացուցեալ լցին զփափագ ﬔր և զհեշտութիւն․ վասն որոյ առաւել յայսմ հետէ պաշտելիք են ի մէնջ և փառաւորեալք, իբրև հեշտացուցիչք և կամակատարք։» Դարձեալ քրիստոնեայ մատենագրի համար ﬕ յարմար առիթ էր այդ առասպելի մէջ բացատրելու նաև Առլեզների պաշտամունքի խարէական ծագումը․ «Կանդնէ և \emph{նոր իﬓ պատկեր} յանուն դիւաց, և ﬔծապէս ղոհիւք պատուէ․ ցուցանելով աﬔնեցուն, իբր թէ \emph{այս զօրութիւն} աստուածոցն կենդանացուցին զԱրա։» Այս գրուածքը մութն է․ «նոր իﬓ պատկեր» բառը կարելի է հասկանալ, թէ Շաﬕրամը նոր աստուածութիւն չէ հաստատում, այլ եղած աստուածների համար «ﬕ նոր պատկեր, արձան» է կանգնում։ Բայց կարելի է և այնպէս հասկանալ թէ նա նոր աստուածութիւն, Առլեզների պաշտամունքն է հաստատում։ Այն ժամանակ այդ հակասում է նոյն իսկ առասպելի էութեանը, քանի որ հայոց մէջ այդ հաւատքի գոյութիւն ունենալն արդէն ենթադրւում է, որ հայերը հաւատում են թէ Արան կարող է յարութիւն առնել և յարութիւն է առել։ Հակասութիւնն իսկ ցոյց է տալիս, որ քրիստոնեայ մատենագրի կողﬕց է աւելացրած այդ։ Այդ հատուածն Անանունի մէջ (Սէբէոս, եր․ 6․) դտնուﬔնք այսպէս․ «Եւ այնպէս հանէ համբաւ Արալեզաց տիկինն Շաﬕրամ,» որ դարձեալ մութն է։

Ինչպէս էլ լինի, այդ գծերը պարզ կերպով քրիստոնեայ հեղինակից են մտած հին առասպելի մէջ, որ քրիսաոնէական ձևով և հայեացքով է պատմուած։ «Եթէ այդ և հեﬔրական յաւելուﬓերը դէն դնենք, ղրում է պ․ Գելցերն այս առասպելի համար, կըﬓայ ﬕայն ասորական սիրոյ աստուածուհու և իւր տարփածուի առասպելը։ Շաﬕրաﬕ զրոյցի մէջ պահուած է Աստղկի առասպելը,» աւելացնում է նա։\footnote{Zur armen. Götterlehre. եր․ 132։}

Թէ Արան ժողովրդական զրոյցով յարութիւն է առել, այդ աեսնում ենք և Թոմա Արծրունու մէջ։ Այստեղ (եր․ 215) գտնում ենք ﬕ այսպիսի աղճատուած կաոր։ «Եւ ղաշիկ արարեալ յայնկոյս քան զՎանաոսպ, ի տեղւոջն արձանաշար քարակարկառ գոգաձև ﬕջոցի երկուց բլրակաց, որ հայի յերիվարաց արկման դաշտն, ի վերայ Լեզուոյ գեաւղջն, \emph{որ զօրացն գեղեցիկ առասպելաբանեն սպիանած վերայն սպանելոցն ի մանկանցն Շաﬕրամայ}։» Վերջին ընդգծած մասը աղաւազուած է։ Այսքանը ﬕայն հասկացւում է աղճատուած ձևից թէ Շաﬕրաﬕ մանուկներից ըսպանուածների վէրքերի սպիանալու ﬕ առասպել է դա։ Բայց բնագիրը վերականգնելը շատ հեշտ է։ Աղճատումը ﬕայն տառերի՝ յ, ց, ե, ի ևլն մէջ է։ Պատկանեանն արդէն սկզբի երեք բառի համար իբրև ծանօթութիւն դընում է․ «թերևս՝ \emph{ուր զԱրայն գեղեցիկ}։ (տես․ նոր․ Բուզ․ յէջն 20)։ Այնուհետև «սպիանած» բառը շատ պարզաբար կամ աշխարհաբառի ձևով «սպիացած,» կամ աւելի «սպիանալ» պէտք է կարդալ։ Անհասկանալի «վերայն» բառը «վիրացն» է, քանի որ վէրքերը պէտք է սպիանան։ Այսպէս բնագիրը կըստանանք․ «Ուր զԱրայն գեղեցիկ առասպելաբանեն՝ ըսպիանալ վիրացն սպանելոյն ի մանկանցն Շաﬕրամայ։» Մենք սպանելոցն, բառն ևս եզակի «սպանելոյն» դարձրինք (Հմմտ․ Խորենացի․ Ա․ ժե․ ﬔռանի Արայ ի պատերազﬕն ի մանկանցն Շաﬕրամայ)․ ﬕ անգամ որ «գարայն» դարձած է «զաւրացն,» ետևից պետի բերէր և «սպանելոցն» յոգնակի թիւը։ Թովմայի այս կտորից իմանում ենք, թէ առասպելաբանում են, որ Շաﬕրաﬕ մանուկներից սպանուած Արայ գեղեցկի վէրքերը սպիանում են, ուստի և Արան կենդանանում է, Լեզուոյ գիւղում։

Նոյն զրոյցը գտնում ենք և այսօր։ Ներսէս Սարգիսեանն իւր Տեղագրութեանց մէջ գրում է նոյն գիւղի մասին․ «Ջորմէ ասի յոմանց տեղի լինել անկմանն Արայի ի Շաﬕրամայ, \emph{որոյ զոﬓ ի համանեացն զԱրայ կարծեցուցեալ՝ լուր եհան եթէ դին լիզեալ ողջացուցեալ են}․ և յայնմանէ, ասեն, ﬓաց անուն գեղջն Լէզք։» \footnote{Ն․ Սարգիսեան, Տեղագրութիւնք ի Փոքր եւ ի Մեծ Հայս, Վենետիկ․ 1864․ եր 264․} Ինչքան էլ Սարգիսեանը ժողովրդից է առնում, ինչպէս պարզ երևում է «ասի յոմանց,» «ասեն» բառերից, բայց նա ժողովրդականը հաւատարմութեամբ արձանագրել չէ կարողանում ընդ դծած տողերի մէջ։ Նա ենթարկւում է Խորենացուն։\footnote{Հմմտ․ Խորենացի Ա․ ժե․ «Ջﬕ ոﬓ ի հոմանեաց իւրոց զարդարեաշ․․․ համբաւե զնմանեայսպես, շիզեաշ աստուածոցն զԱրայ եւ կենղանացուցեալ։» Հմմտ․ Անանուն (Սեբեոս եր․ 5) «Ջարդարե զﬕ ոﬓ ի հոմանեաց իւրոց այր պատշաճող, եւ համբաւ նանե զԱրայի շիզուշ աստուածոցն եւ յարուցանել։»} Աւելի հաւատարմութեամբ գրի է առնում նոյն զրոյցը Սրուանձտեանը Գրոց Բրոցի մէջ (եր․ 52)։ Լէղքը իւր անունով և հին պատմութեամբ յայտնի է․ ուր Շաﬕրաﬕ աստուածները ﬔր \emph{Արայ գեղեցկին ﬔռած մարﬕնը լզեր ու կենդանացուցեր են} եղեր․ և Արայլէզք անունով կուոք ու աստուած շինեցին պաշտեցին այն ժամանակ հայեր այս գեղի բարձր գագաթին վրայ, ուր այժմ Աﬔնափրկչի մատուռն է։\footnote{Թէ ուﬓից ե առնում Սրուանմտեանն այս զրոյցը՝ յայտնի չե․ եբե արդի ժողովրդական զրոյցները չեն աղբիւրը այն ժամանակ կարող երնք ենթադրել թէ նա Թ․ Ածրունուց ե առնում եւ թէ նորա ﬔռին եղեշե ﬕ ﬔռացիր, որի ﬔշ Արայի վերքերի սպիանալու կտորն աղճատուած չե եղել։ Դժուար ե ենթադրել, որ Սրուանմտեանը կը վերականգներ այդ աղճատուած կտորը։ Սրուանմտեանի դիրքը տաոն երեք տարի կաղ ե նրատարակուած քան Պատկանեանի՝ Թոմայի հրատարակութիւնը։}

Այս պատմութեան էութիւնը, Լէզք գիւղի հետ կապուած այս առասպելը, թէ Շաﬕրաﬕ մանուկներից սպանուած Արայի վէրքերն այդտեղ են սպիանում, Խորենացուց չէ ծագում ոչ Արծրունու և ոչ ﬕւսների մէջ, գի ըստ Խորենացու նախ՝ Արան յարութիւն չէ առնում և ապա նա սպանւում է Արայի դաշտուﬓ, Այրարատում, և այնտեղ էլ Շաﬕրամը հրամայում է աստուածներին լիզել նորա վէրքերը։ Եւ երբ նա հաւատացնում է հայերին թէ Արան յարութիւն է առել, նոր գալիս է Վանայ կողﬔրը (Ա․ ժգ․)։

Ասել թէ Արծրունին և Սրուանձաեանն իրենցից ստեղծում են այդ զրոյցը, ելակէտ ունենալով ﬕայն Լէզք անունը, այդ հեշտ պրծնելու ﬕ ﬕջոց է ﬕայն։ Եւ այդ հեշտ ու պատրաստ, ﬕջոցն է բանեցնում պ․ Խալաթեանը խեղճ Թովմա Արծրունու վերաբերմամբ ևս։ Սորա պատմածները «Շաﬕրաﬕ և Արայի, ինչպէս և ﬕ քանի հայ նախարարների և թագաւորների մասին, ինչպէս են Հայկ, Տիգրան Ա․, Վահագն, Երուանդ, Արտաշէս Բ․ և ուրիշները, բնաւ պէտք չէ համարել, ինչպէս կարող է երևալ, Խորենացու բերած վիպական հատուածների վարիանտներ, և այս հիման վրայ այս վերջիններիս իսկութեան հոմար ապացոյցներ կազﬔլ։ Միանգամ ընդ ﬕշտ պէտք է իմանալ, յարում է կարուկ կերպով պ․ Խալաթեանը, որ Թոմասը իւր պատմութեան հին մասը գրելիս \emph{սովորաբար} (այո՛, բայց ﬕայն սովորաբար) կրկնում է Խորենացու ասածը, կամ թէ \emph{մշակում է նորա գրածը ոչ առանց ֆանտաստիկական յաւելուﬓերի}, որոնց նպատակն է ﬔծացնել Արծրունիներին։» Եւ այդ աﬔնն այնքան ակնյայտնի է, աւելացնում է պ․ Խալաթեանը, որ նոյն իսկ ժամանակի կորուստ է» այդ քննելը։\footnote{Халат. Арм. Эпосъ. եր․ 151․} — Եւ այսպէս վերջացաւ․ «սովորաբար» Խորենացի և «ֆանտաստիկական յաւելուﬓեր» և «ժամանկի կորուստ» այդ քննելը։ Թ․ Արծրունին էլ գըտաւ իւր դատավճիռն ու գնահատութիւնը։ Խորենացու ﬔղքը ﬕշտ այն է, որ նա շատ բան պատմում է, որ իրենից առաջ ուրիշները չեն պատﬔլ։ Իսկ Թովմա Արծրունու ﬔղքն էլ այն է, որ նա Խորենացուց յետոյ է ապրել և Խորենացուց օդաուել է, թէպէտ և կան բաներ, որ նորանից տարբեր է պատմում, ինչպէս այս Լեզուոյ գիւղի զրոյցը։ Այդ գէպքում նա «ֆանտաստիկական յաւելուﬓեր» է անում։

Բայց Թոմա Արծրունին, օրինակ, Տիգրանի և Արտաշէսի պատմութիւնն անելիս իբրև աղրիւր դնում է յոգնակի թուով․ «Որպէս ցուցանեն \emph{առաջինքն ի պատմագրաց} (եր․ 36), «Որպէ բացայտյտեն \emph{զիրք պատմագրացըն} (եր․ 52)։ Այդ սուտ է դրում նա․ ﬕայն ﬕ աղբիւր ունի, Խորենացի․ ﬓացածը ֆանտասաիկական յաւելուﬓեր են։ Ուրիշ տեղ (եր․ 44) նա գրում է․ «Իսկ առ ﬔզ հասին զրոյցքս այս ըստ ﬓացորդաց \emph{պատմագրացն առաջնոց}, ի Մամբրէէ վերծանողէ և ի նորուն եղբօրէ Մովսէս կոչեցելոյ, և ﬕւսուﬓ Թէոդորոս Քերթող․․․։» Այստեղ էլ նա սուտ է գրում։ Կամ երբ Հայկի և Բէլի պատերազﬕ մասին գրելիս աւելացնում է․ «Ջոր (զԲէլ) \emph{ոմանք ի պատմագրաց} ասեն փախստեամբ դարձեալ գնաց յԱսորեստան» (եր․ 24), — այստեղ էլ նա «ֆանտաստիկական յաւելում» է անում, որովհետև Խորենացին այսպիսի բան չէ յիշում։ Կամ եըբ Թոմա Արծրունու մէջ Հայկը Բէլին ասում է, «Շուն դու, և երամակ շանց՝ որ զկնի քո տողեալ սահին,» այստեզ էլ «ֆանտաստիկական» յաւելում է անում, զի այդ չկայ Խորենացու մէջ․․․ բայց չէ՛, այդ «ֆանտազիա» չէ, զի բարեբախտաբար ﬓացել է Անանումը (Սերէոս, եր․ 4), որի մէջ կայ «Շուն ես դու և յերամակէ շանց՝ դու և ժողովուրդ քո։» — Այդպէս հեշտ կերպով կարելի չէ պրծնել ﬕ պատմագրի համար, որ աղբիւրներ է յիշում, և որ ինքն իւր համար ասում է թէ ձեռնարկելով գրել «Որ ինչ իսկզբանց անտի շինութեանց աշխարհիս Հայոց ի Հայկայ աղեղնաւորէ և ի նորուն զարﬕց և ցկաթոտ և վավաշոտ Շաﬕրաﬓ տիկին Ասորեստանեայց \emph{եւ ի նմանէ յայլս եւ յայլոց եւս յաւետ ձեռակերտք եւ շինուածք յիւրաքանչիւրոցն յերկրիս ﬔրում եղեն՝ ﬔր առ աﬔնայնն անձամք հասեալ եւ աչօք տեսեալ}․․․» (եր․ 293)։ Թոմա Արծրունին Խորենացուց, անշուշտ, խիստ շատ է օդտւում․ բայց նա ունի և ուրիշ գրաւոր աղբիւրներ։ Խնդիրը, հարկաւ, այն չէ, որ Թոմա Արծրունին ևս կարող է այս կամ այն բանն իրենից յարմարեցրած լինել, բայց աﬔն ինչ, որով նա տարբերւում է Խորենացուց, նորա յարմարեցրածը համարել կարելի չէ։ Այդպիսի յերիւրուﬓերը կամաց կամաց և բազմաթիւ մարդոց ձեռով են պատրաստւում։ Եւ երբ ﬔնք ոչ ﬕ փաստ չունինք այս կամ այն պատմուածքի համար թէ Արծրունու իրեն սարքածն է, չենք էլ կարող նորա «ֆանտաստիկական յաւելումը» համարել։ Այսպէս և երբ նա պատմում է թէ առասպելաբանում են, որ Շաﬕրաﬕ մանուկներից սպանուած Արայ գեղեցկի վէրքերն ըսպիացած են Լեզուոյ գիւղում, — ﬔնք իրաւունք չունինք ասելու թէ նորա աղբիւրը Խորենացին և իւր «ֆանտազիան» են։ Նորա աղբիւրն այդտեղ ժողովրդական զրոյցն է։ Մենք այդ կըտեսնենք ներքևում ևս։ Այստեղ աւելորդ չի լինիլ ﬕ երկու խօսքով կանգ առնենք պ․ Խալաթեանի ﬕ ուրիշ կարծիքի մասին ևս։

«Արայի զրոյցն արուեստական կերպով շինուած է \emph{ﬕ ամբողջ շարք գրաւոր յիշատակարանների վրայ},» գրում է պ․ Խալաթեանը (եր․ 145)։ Եւ այդ «գրաւոր յիշատակարաններն» են, Դիոդոր Սիկիլիացի, Աստուածաշունչ, Եզնիկ, Փաւստոս, Կղեմէս Աղէքսանդրացի։ Թէ Խորենացին կամ իւր աղբիւրը՝ Մար Աբաս Շաﬕրաﬕ պատմութեան համար կարող էր օգտուել այս կամ այն գրաւոր յիշատակարաններից և օգտուած էլ է, այդ ﬕ կողﬓ ենք թողնում։ Բայց շատ զարմանալի է, որ այդ զրոյցը «\emph{սարքողը},» ըստ պ․ Խալաթեանի, ﬕ կտոր Դիոդոր Սիկիլիացուց է առել, ﬕ կտոր Աստուածաշնչից, ﬕ կտոր Եզնիկից, ﬕ կտոր Փաւստոսից, ﬕ ուրիշ կտոր էլ Կղեմէս Աղեքսանդրացուց, ﬕ բառ էլ «Արա» (= գեղեցիկ) պարսկերէնից է առել, ﬓացածն էլ «իւր հարուստ երևակայութեամբ» լրացրել է, և կամ «բիբլիական — քրիսաոնէական հայեացքներով» առաջնորդուած՝ Յովսէփ գեղեցկի պատմութեան ձևով «ողջախոհութեան զգացմունքից» Արային ﬔրժել է տուել վաւաշ Շաﬕրաﬕ սէրը\footnote{А Халат. Арм. Эпосъ. եր․ 143 հտն․} (ﬕ ողջախոհութիւն, որ հայերն անպատճառ բիբլիական քրիստոնէականից պիտի սովորէին), — և ահա դուրս է եկել ﬔր Շաﬕրաﬕ և Արայ գեղեցկի առասպելը պատահաբար ճիշտ նոյնը, ինչ որ Շաﬕրամ — Իշտարի և իւր գեղեցիկ ու ողջախոհ սիրական Իզդուբարի առասպելն է, ﬕ առասպել, որ սեպագրութեանց մէջ կարդացուած է։ Պ․ Խալաթեանի ճանապարհը շատ երկար ու պատահականութեամբ լի է։ Կարծում ենք կարճ ու բնական ճանապարհն այն է, որ ասենք թէ քանի որ սեմականների սիրոյ վաւաշ, բուռն զգայական դիցուհու պաշտամունքը մտած էր ﬔր մէջ, ուստի և այդ առասպելն ևս, որ նոյն դիցուհու առասպելն է, կարող էր դիցուհու պաշտամունքի հետ մտնել հայոց մէջ։ Այս կասէր անշուշտ պ․ Խալաթեանն, եթէ այդ առասպելն աﬔնից առաջ յիշուած լինէր ուրիշ աﬔն ﬕ մատենագրի, օրինակ՝ Փաւստոսի, Եզնիկի մէջ, բացի Խորենացուց։ Եւ Խորենացին ﬔղաւոր է, հարկաւ, որ ինքն է աﬔնից առաջ յիշում այդ։ Բայց ﬔնք դառնանք Վանայ Լէզք կամ, ըստ Թոմայի, Լեզուոյ գիւղին։

\emph{Լէզքի եւ Արտաﬔտի զրոյցները}։ — Արուանձտեանը վերևում յիշուած գրքի մէջ ասում է թէ Լէզք գիւղի բարձր գագա թի վրայ, ուր այժմ Աﬔնափրկչի մատուոն է, «\emph{բեւեռագրով} քարեր լեցուն են․ և \emph{ձուլաձոյ հին պղնձէ կենդանեաց փոքր արձաններ} գըանուեցան ﬔր օրերը դեռ ևս քանի ﬕ տարի առաջ։» Այս նկարագիրը ցոյց է տալիս, որ Լէզքը շատ հին ժամանակ արդէն, Արծրունուց և Խորենացուց շատ առաջ ﬕ դեր է խաղացել։ Դա հաւանօրէն եղել է կրօնական պաշտամունքի տեղ, որի հեաքերը ﬕնչև այժըմ ﬓում են․ «Այս Լէզք գիւղի մօտերը կայ ﬕ թոնիրի ձևով \emph{աղբիւրի փոսակ} կամ ջրհոր, որ \emph{Սուրբ թոնիր} կըկոչեն, ուր յաճախ գնացողներ կան, շատերուն իբրև \emph{ուխտատեղի}։» Հեթանոսական պաշտամունքի ﬓացորդը լինելը շատ պարզ է։ Բայց ի՞նչ պաշտամունքի ﬓացորդ է այդ։ Շաﬕրաﬕ — Իշտարի — Աստղկի։ Այս բացատրում է ﬔզ «Սուրբ թոնրի» հետ կապուած առասպելը, որ բերում է Սրուանձտեանը։ «Մի ﬕտյն մէկ ձուկ կերևի այդ ջուրին մէջ․ և \emph{որին որ երեւի, անոր բախտն ու ուխտը կը կատարուի}։ Դա \emph{կնկան կերպարանք} ունի եղեր․ և արծաթէ օղն ալ իւր քիթն անցուցած՝ կըտեսնուի տակաւին։ \emph{Երէցկին} է եղեր դա․ \emph{կին կարի գեղեցիկ}․ երբ նստած թոնրի շուրթն հաց կըթխէ, աղքատ մը կուգայ հաց կ՚ուզէ, կուտայ։ Կերակուր կուզէ, կուտայ։ Գինի կ՚ուզէ, կուտայ։ Թշուառականը կըհամարձակի \emph{պագ} ﬕ ուզելու, երէցկինը կըվարանի, բայց․․․ \emph{պագն ալ կուտայ}։ Եւ յանկարծ նոյն վայրկենին երէցը ներս կըմտնէ․ երէցկինը ամօթէն ու ահէն ինքզինքն կըձգէ ի թոնիր կրակին մէջ․․․ կրակն ջուր \emph{դառնալով, ինքն ալ ձուկ}, Աստուծոյ հրամանով յաւիտենական յիշատակ կըﬓայ նոյն տեղ։»\footnote{Սրուանմտեան, Գրոց Բրոց, եր․ 53․ Այս առասպելի վարիանտը, ոչ տեղի հետ չկապուած, Տ․ Նաւասարդեանի Հայ ժող․ հեքեաբներում, Է․ գիրք․ եր․ 39․}

Այս զրոյցն ինքը պ․ Խալաթեանը համարում է, նոյն իսկ իւր «Հայոց վէպի» մէջ, Շաﬕրաﬕ առասպելի շրջանին պատկանող ﬕ պատմուածք, որ «\emph{թերեւս ի հնուց անտի Հայաստան է մտած Ասորիքից} և ժամանակի ընթացքում կորցրել է իւր արտաքին, հեթանոսական պարական, բայց ոչ ֆաբուլան։ Այս զրոյցի հնագսյն ձևը բերած է Դիոդոր Սիկիլիացին․ Դա Դերկետոյ դիցուհու զրոյցն է, որ իւր դուստր Շաﬕրաﬕ ծնուելուց յեաոյ ձուկ է դառնում յանցաւոր սիրոյ համար։\footnote{Халат. Арм. Эпосъ., եր․ 150․} Պ․ Խալաթեանն այս մասին խօսած է իւր ﬕ ուրիշ աշխատութեան մէջ,\footnote{Очеркъ народн. арм. сказокъ, Москва, 1885.} որ ցաւ է ձեոի տակ չունինք։ Մենք չենք յիշում թէ այս տեղ նա ինչպէս է հաստատում իւր այդ կարծիքը, բայց ﬔնք այդ ճիշտ ենք համարում։

Շաﬕրաﬕ մայրը Դերկեաոն (Atargatis) ջրի աստուածուհի է։ Նորա սրբազան կենդանիքն են Փիւնիկիայում և Ասորիքում աղաւնին և ձուկը։ Ասկաղոնում, ուր նորա պատկերը ձկան մարﬓով էր հանած, նորա տաճտրը շինուած էր ձկնահարուստ ﬕ լճի մօտ։ Նորա տօնի օրերը ջուր էին բերում նոբա տաճարը և սրսկում։ Այս աﬔնը յիշեցնում է ﬔր Վարդավառին, որ Անահտի կամ Աստղկի տօնն է համարւում, ջուր սրսկելը և աղաւնի բաց թողնելը, ﬕ սովորութիւն, որից երևում է, որ ﬔզնում ևս այդ աստուածուհուն նուիրական եղած են աղաւնին և ջուրը։ Պ․ Խալաթեանը կարծում է թէ Լէզքի այդ առասպելը կորցրել է իւր հեթանոսական պարագան։ Բայց այդ ճիշտ չէ։ Սուրբ թոնրի, \emph{ջրի ու ձկան} պաշտամունքն իսկապէս հեթանոսական է և ոչինչ չունի քրիստոնէութեան հետ։\footnote{Թոնիրն ու թոնրում այրուելը բոլորովին ուրիշ մոտիւ ե, որ եկել կապուել ե սեմական սիրոյ ղիցուհու այս առասպելին։ «Սուրբ թոնիրը» գուցե հեթանոսական օչախի պաշտամունքի ﬓացորղ լինի։ Հմմտ․ Գղ․ Ե․ Հայկի առասպելը․} Այս պաշտամունքը հարկաւ նոր չէ և գալիս է շատ հին հեթանոսական ժամանակներից․ նոր չէ և այդ Սուրբ թոնրի հետ կապուած առասպելը, որ սիրոյ աստուածուհու այս պաշտամունքի հետ ի հնուցանտի մտած պիտի լինի Հայոց մէջ, իսկապէս ոչ թէ մտած, այլ հայացած ասորիները պահած են այդ կաղﬔրում։

Ի՞նչ է եզրակացութիւնը։ Այսօր Վանայ նոյն Լէզք գիւզի մօտ, ուր ըստ Արծրունու սպիանում են Շաﬕրաﬕ մանուկներից սպանուած Արայի վէրքերը, գտնում ենք սեմականների սիրոյ դիցուհու և հեթանոսական պաշտամունքը․ և որ գլխաւորն է, այդ պաշտամունքի տեղի հետ կապուած է, թէպէտ և առանց Շաﬕրամ անուան, ﬕ զրոյց, որ քըննութիւնը ցոյց է տալիս թէ նոյն սիրոյ աստուածուհունն է։ Այսօրուայ այդ զրոյցն ու պաշտամունքը չկան Խորենացու և Արծրունու մէջ և չեն էլ կարող սոցանից անցած լինել ժողովրդին։ Դա հնութեան ﬓացորդ է։ Եւ երբ այսօր Խորենացուց անկախ գտնում ենք սեմական սիրոյ աստուածուհու պաշտամունքն ու առասպելը Վանայ մօտ Լէզք գիւղում, ﬕթէ դա ապացո՞յց չէ, որ այդ տեղը հնումը նուիրական է եղել Շաﬕրաﬕն։ Եւ ﬕթէ այսօրուայ այդ պաշտամունքն ու զրոյցը, որ շինծու չեն, չե՞ն վկայում, որ նոյն տեղում կամ Վանայ մօտիկ ուրիշ աեղերում արդարև եղել են Շաﬕրաﬕ վերաբերեալ ուրիշ զրոյցներ ևս։ Եւ եղել են այդ զրոյցները ու կան ցայժմ, որոնցից մէկն է Արծրունու գրածը, այսինքն Արայ գեղեցկի սպանուելը Շաﬕրաﬕ մանուկներից և նորա յարութիւն առնելը․ ﬕ ուրիշն է \emph{Շաﬕրաﬕ} մահուան մասին Խորենացու պատմածը․ ﬕ երրորդն է արդի զրոյցը Շաﬕրաﬕ մասին, որ բերում է Ն․ Սարգսեանն իւր Տեղագրութեանց մէջ։

Լէզք գիւղի հետ կապուած այսօրուայ այդ ﬕ զրոյցը ﬕայն, որ Խորենացուց անկախ գոյութիւն ունի, ցոյց է տալիս որ ﬔնք իրաւունք չունինք Շաﬕրաﬕ մասին պատմուած արդի ﬕւս զրոյցը «Խորենացու ազդեցութեամբ կազմուած ﬕ նորագոյն զրոյց» համարելու, ինչպէս կարծում է պ․ Խալաթեանը (եր․ 149)։ Նորա պատճառաբանութիւններից գլխաւորն այն է, թէ «գործողութեան տեղը ﬕևնոյն է, Վանայ լճի շրջակայքը։» Սակայն այդ կարող չէ հաստատել թէ այսօրուայ Շաﬕրաﬕ մասին զրոյցները Խորենացուց են անցած ժողովրդին։ Զէ՞ որ Լէզք գիւղի այդ Սուրբ թոնրի առասպելի համար ևս գործողութեան նոյն տեղն է․ բայց այդ առասպելը Խորենացուց չէ մտած ժողովրդի մէջ։ Գործողութեան տեղի նոյն լինելը կարող է ﬕնչև իսկ պ․ Խալաթեանի կարծիքի հակառակը հաստատել։ Եթէ ﬕ հին զրոյց յարատևում է ժողովրդի մէջ, ասել չի ուզիլ աﬔնից աւելի այնտեղ պիտի յարատևի, ուր առասպելը կապուած է որոշ տեղի հետ։ Եթէ հնումը Շաﬕրաﬕ զըրոյցները Վանայ մօտերն են այս կամ այն տեղի հետ կապուած, շատ բնականօրէն ﬔնք այսօր ևս նորա առասպելներն աւելի շուտ նոյն տեղերում կըգտնենք քան որև է ուրիշ տեղում։

Խորենացին չէ ասում թէ Շաﬕրաﬕ մահն իւր իմացած զրոյցով յատկապէս ո՛ր տեղի հետ է կապուած։ Նա ﬕայն ընդհանուր ձևով գրում է․ «Աստ ուրեﬓ գմահն ասել Շաﬕրամայ։» Ենթ ադրաբար իմանում ենք Վանայ կողﬔրը։ Իսկ արդի զրոյցը որոշ կերպով ցոյց է տալիս այդ տեղը․ Շաﬕրաﬕ վերջը կապուած է Արտաﬔտի հետ։ Եւ այս իւր առանձին նշանակութիւնն է ստանում, երբ գիտենք, որ ըստ Թոմա Արծրունու (եր․ 53 հտ․) նոյն տեղում եղել է Աստղկի պաշտամունքը։

Այնուհետև պ․ Խալաթեանի երկրորդ պատճառաբանութիւնն է թէ Սարգիսեանի բերած արդի զրոյցի մէջ ևս «վհուկ և հեշտասէր» է դուրս գալի Շաﬕրամ ինչպէս Խորենացու մէջ։ Սխալւում է պ․ Խալաթեանն ասելով թէ այդ գծերը «Խորենացու իւրատեասկ պատկերացուﬓ են Շաﬕրաﬕ մասին» (своеобразное представленые)։ Շաﬕրաﬕ հեշտասիրութիւնն ու տռփանքը, հոմանիներ որոնելը գիտենք, Խորենացու ստեղծածը չէ։ Իսկ ինչ որ վերաբերում է Շաﬕրաﬕ վհուկ լինելուն Խորենացու մէջ, իսկապէս նորա յուռութներին ինչպէս և ու լունքները ծովը ձգուելուն, նոյն իսկ պ․ Խալաթեանը կարծում է թէ այդ կարող է լինել «Խորենացու ժամանակների յետնագոյն (?) զրոյցների արձագանգներ» (եր․ 148 հտ․)։\footnote{Իսկապես չենք հաոկանում «Խորենացու ժամամանակների յետնագոյն զրոյցների արձազանգներ» (отго лоски позднихъ сказаный) բաոերով ինչ ե ուզում ասել պ․ Խալաթեանը։ Խնղիրը նորա համար այն չե, թէ այդ զրոյցները Խորենացուց 1500-2000 տարի առաշ են ատեղծուած, թէ ﬕ քանի հարիւր տարի ﬕայն առասջ․ այլ այն թէ եղել են Խորենացու ժամանակ զրոյցներ Շաﬕրաﬕ մասին թէ չե, և Խորենացին այդ զրոյցներից ե օզտուել թէ չե շինին զրոյցները հնագոյն թէ ուշ ժամտնակներում ստեղծուած։} Եւ երբ ﬕ անգան Խորենացու ստեղծածը չեն այդ գծերը, ուստի և ոչ ﬕ արժէք չունի այն թէ Խորենացու մէջ և արդի զրոյցի մէջ Շաﬕրամը վհուկ ու հեշտասէր է դուրս գալիս։ Այդ կընշանակէ ﬕայն, որ նոր զրոյցի մէջ Շաﬕրամը, ինչպէս իւր անումը, նոյնպէս և իւր բնաւորութեան այդ երկու էական գծերը պահել է։ Բայց այդքան ﬕայն։ Էլ ուրիշ ոչ ﬕ նմանութիւն չենք գտնում Շաﬕրաﬕ մասին Խորենացու պատմածի և ﬔր արդի զրոյցի մէջ, որ բերում է Սարգիսեանը։ Ի զուր է կարծում պ․ Խալաթեանը թէ Խորենացու Ջրադաշտի և արդի զրոյցի ծերունու մէջ ﬕ հաﬔմատութիւն է գտել։ Այդ երևակայական է ﬕայն։

Բաղդատութեան համար ﬔնք այստեղ կրրերենք Արտաﬔտի զրոյցը, որի բնագիրը գտնելը դժուար է։ Նկատենք առաջուց, որ ﬔնք չենք վստահանում ասելու թէ Սարգիսեանը հաւատարմութեանբ գրի առած լինի ժողովրդական զրոյցը, առանց այս կամ այն բանի մէջ ենթարկուելու Խորենտցուն։ Լէզքի զրոյցի մէջ նա ցոյց է տուել, որ ինքը հարազատ գրի առնող չէ։

Լուիցուք և զﬕւս առասպելն զոր ասեն զՇաﬕրամայ, զոր և համառօտ բանիւ յիշէ Խորենացի ասելով Ուլունքի ծով Շաﬕրամայ։ Թագուհին Շաﬕրամ որ ﬕ ի հզօր և ի ﬔծագործ ինքնակալացն էր, շրջագայեալ, ասեն, երբեﬓ զբօսանաց աղագաւ յերկրին Վասպուրականի՝ տեսանէ ուրեք մանկունս խմբեալ ի ﬕ․ և ﬔրձ եղեալ առ նոսա տեսանէ զի գտեալ նոցա ուլունս ի գետնի՝ ի զնին կան նոցին․ իսկոյն ծանուցեալ զպատուականութիւն նոցա՝ առնու ի ձեռաց մանկանցն և պարգևս տուեալ արձակէ զնոսա։ Ուլամրքն այնոքիւք սկսանի Շաﬕրամ մոգել և կախարդել, և ըստ ապականեալ սրտի իւրոյ զաﬔնայն չարութիւն գործել ընդ երկիրն համօրէն․ զոր կամէր առ ինքն կոչել ի լրուﬓ տռփանացն՝ զօրութեամբ ուլանցն ոչինչ դըժուարէր, և զոր կամէր կորուսանել՝ և յայնմ յաջողէր անաշխատ, ﬕնչև սասանել աﬔնայն ուﬔք և չկարել ճիկ հանել։ Ծերոյ ուրեﬓ որ երթևեկէր առ նա և աﬔնայն իրաց նորա հմուտ և իբր խորհրդական էր, բազում ժամանակս խորհելով ի մտի թէ զիարդ զերծուսցէ զերկիրն ի ձեռաց նորա կամ ի զօրութենէ ուլանցն, յաւուր ﬕում ﬕնչդեռ յԱրտաﬔտ քաղաքին էր ընդ թագուհւոյն, ժամ դիպող գտեալ՝ յափշտակեալ ի ձեռաց նորտ գուլունս փախչի։ Ջայրացեալ Շաﬕրամայ և ի ցասման ﬔծի եղեալ յարձակի զհետ նորա, և չկարացեալ հասանել՝ առ անհնարին կատաղութեանն զգէսսն պարսատիկ գործեալ․ քանզի յոյժ երկայն և թաւ էին, և ապառաժ ﬕ ահեղ ﬔծութեամբ եդեալ ի նմա արձակէ զհետ ծերոյն․ և առ ﬔծի բռնութեան զերծեալ հերացն ի գլխոյն՝ և ապառաժն ﬔծ գլորեալ անկանի ի փոս ﬕ առ Արտաﬔտաւ, զոր և ցուցանեն ﬕնչև ցայսօր։ Իսկ ծերն առեալ զուլունսն փախչի ի ծովեզերս Դատվանայ և անդ արկանէ ի ծով, և զերծանի երկիրն առ հասարակ ի մոգական չարութենէն Շաﬕրամայ։ Ահաւասիկ զրոյցքն «Ուլունք ի ծով Շաﬕրամայ։»

Խորենացու պատմուածքի և այս զրոյցի նմանութիւնն այն է, որ երկսի մէջ Շաﬕրամը ﬕ վաւաշ ու չար վհուկ է, որ ունի ուլունքներ, որոնք վերջը ծոﬖ են ընկնում։ Սնացածն ամբողջապէս տարբեր է։ Խորենացու մէջ ոչ ﬕ բան չկայ այն մասին թէ ինչպէս է Շաﬕրաﬓ ուլունքները ձեռք բերում․ և այս կազմում է արդի զրոյցի առաջին մասը։ Խորենացու մէջ չկայ և զրոյցի երկրորդ մասը թէ ուլունքները ինչ զօրութիւն ունէին և թէ Շաﬕրաﬓ ինչպէս իւր կիրքն էր կատարում անոնց զօրութեամբ։ Ըստ Խորենացու Շաﬕրամը հոմանիներ է ձեռք բերում ոչ թէ ուլունքներով, այլ նոցա «պարգևելով գաﬔնայն իշխանութիւնս և զգանձս,» կամ «զկամս ցանկութեան» կատարել է տալիս, «ընծայիւք և պատարագօք, բազում աղաչանօք և խոստմամբ պարգևաց,» ինչպէս Արայի զըրոյցի մէջ է, — ﬕ բան որ անում է և Իշտարը։ Մենք չգիտենք՝ Խորենացու մէջ յուռութք — ուլունքներն ինչի են պէտք գալիս։ Այնուհեան զրոյցի երրորդ մասն ևս չունի Խորենացին։ Ջրադաջաը Ասորեստանի կողﬓապետն է, որին հաւատում է Շաﬕրամը իւր իշխանութիւնը, այսինքն կառավարութիւնը նորսն է թողնում։ Նա ուզում է բռնանալ ի վերայ աﬔնայնի։ Շաﬕրամ կռւում է նորա հետ, յաղթւում է, փտխչում է Հայաստան, ուր ամառներն անց էր կացնում, և այստեղ որդին սպանում է նորան։ Մեռնելուց առաջ, ըստ առասպելին, նա ուլուքները ձգում է ծովը։ Իսկ զրոյցի ծերունին երթևեկութիւն ունի Շաﬕրաﬕ մօտ․ «աﬔնայն իրաց նորա հմուտ և իբր խորհրդական էր․» այսինքն գիտէր, որ Շաﬕրաﬓ ուլունքների զօրութեամբ է չարութիւն գործում։ Բարի ծերունին ժողովրդին ազատելու համար՝ խլում է ուլունքները ու փախչում։ Շաﬕրաﬓ ետևից ընկնում է․ մազերը պարսատիկ շինում․ և պոկում։ Եւ այսքան ﬕայն։ Ջրոյցից ﬔնք ﬕնչև անդամ որոշ չգիտենք թէ արդեօք Շաﬕրամը ﬔռնում է թէ չէ։ Միայն նորա ուլունքները ծերունին ծոﬖ է ձգում և ժողովուրդն ազատւում է նորա մոգական զօրութիւնից։ Ասել կարելի չէ թէ այս զրոյցը ստեղծուել է Խորենացու ﬕ խօսքի ազդեցութեան տակ թէ Շաﬕրամ վաւաշ է և թէ «Ուլունք Շաﬕրամայ ի ծով։» Այս նոյնպէս կարող է Արտաﬔտում պատմուած Շաﬕրաﬕ հին առասպելի ﬓացորդ լինել, ինչպէս Լէզքի Սուրբ Թոնրի առասպելն է։ —

Աւելորդ չի լինիլ վերջում նկատել, որ Խորենացու բերած առասպելով Շաﬕրաﬕ վախճանն ուրիշ տեսակ է լինում։ Այստեղ Շաﬕրաﬓ իւր թշնաﬕներից հալածուած փախչում է․ առասպելը պատմում է նորա «զհետևակ փախուստն, և զպասքուﬓ և զիղձըս ջրոյն և զարբուﬓ, այլ և ի մօտ հասանել սուսերաւորացն, և զյուռութսն ի ծով, և բան ի նմանէ՝ ուլունք Շաﬕրամայ ի ծով։ Այլ․․․ և Շաﬕրամ քար։» Շաﬕրաﬓ ուրեﬓ իւր ուլունքները ծովը ձգելուց յետոյ քար է դառնում։

\bigskip

Մ․ Աբեղեան․}
\clearpage

\end{document}
